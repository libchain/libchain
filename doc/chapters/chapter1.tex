\chapter{Introduction\label{cha:introduction}}
LibChain aims to provide decentral, verifiable and anonymous access management based on blockchain technology. LibChain envisions a novel procedure to access digital media from different publishers through a library.  With the LibChain service, the library stores every request of a digital publication directly in the blockchain, making it an anonymous but verifiable source for publishers to provide access to the user. The decentral blockchain architecture enables new access models for digital media and allows fairer and more accurate pricing models based on the usage. 
\\
\\
The key idea in LibChain is that libraries can buy instances of a digital book and can from then on control who has access to it. Publishers are still in control of how the digital media is accessed but can not unmask the identity of the client only whether he has access to a specific book or not. This enables libraries to work closer together and share digital resources just as they would share physical books between partner libraries.
\\
\\
The implementation of our research prototype is based on the open source blockchain framework ethereum. The key advantages of LibChain are: First, it provides a distributed, failure tolerant mechanism for free, open and anonymous access to content that can be used out-of-the-box by anyone. Second, it is compatible to conventional business models, where publishers sell their content for a predefined price. This compatibility is aimed to accelerate the adoption of LibChain by maximising network-external value by integrating the open access and payed publication segments. Third, LibChain provides a reliable way to measure the impact of content and realise payments, since due to its blockchain core, all interactions with LibChain can be verified and trusted. Fourth, LibChain balances the access to information relevant to content providers and the privacy needs of users. Given these advantages, LibChain has the potential to accelerate the shift to an open access publication model that provides a robust, distributed content exchange mechanism and substantially higher utility than conventional models to all involved parties.


\chapter{Motivation\label{cha:motivation}}
Current contracts between academic publishers and research libraries are based on subscription models, granting patrons of a library almost unlimited access to the digital publications of a single publisher. Unfortunately, pricing models often do not correspond to the usage of the publishers content. 
\\
\\
Libraries already cooperate with one another when it comes to physical books in the form of inter-library loans but have not yet tapped into the potential of sharing digital media access between libraries just as they would physical books. As a design principle of the underlying blockchain technology, no mutual trust is required to generate verifiable transactions which expands the number of potential partner libraries immensly.
\\
\\
Verifiable usage metrics among distributed libraries are essential for supporting OpenAccess publications and enabling anonymous compensations and donations will further help make OpenAccess a viable alternative to current models. Conferences and smaller research institutes could be independant from big publishers since LibChain is an open and free platform. Hence, small groups of authors can easily contribute to the LibChain universe and act as their own open access publisher.
\\
\\
\imp{muss noch angepasst werden !!!}
\chapter{Blockchain Technology\label{cha:chapter1}}

A blockchain is a distibuted database that stores informations on linked blocks. To ensure the integrity of the information, every block includes a hash value of its previous block and a Merkel Root of all transactions in the block. Modifications on a confirmed block would lead to a different hashvalue. Therefore, manipulating informations on a blockchain is very hard and provides a high level of reliability.
Informations are stored on the blockchain through transactions. The validity of transactions are verified by miners and are bundeled up to blocks. Miners have to provide a Proof-of-Work in order for their verification of a new block to be accepted by the rest of the network. The chain of blocks with the most amount of Proof-of-Work is where new blocks should be added to. 

\vspace{0.3cm}
\begin{figure}[ht]
  \centering
   \includegraphics[width=\textwidth]{blockchain.jpg}
  	\caption{Simplified Bitcoin Block Chain}
\end{figure}

Blockchains became mainly known in a financial context. The first implementation of a blockchain was used by Bitcoin, a virtual distributed crypto currency.
But in the meantime, blockchain technology has also become popular in other domains i.e. Voting-Systems \cite{yermack2017corporate} or for digital identity management \cite{isaen}. Applications that needs a high level of integrity and reliablilty could benefit from blockchains.

\section{Ethereum}
Ethereum is a public blockchain we used for LibChain, that allows us to implement decentralized applications by means of "Smart Contracts". A smart contract is exectuable code that is stored on the blockchain. It is accessible through a unique address and triggered through transaction calls.
To execute a transaction so called gas is consumed. Gas is the fuel that is being used to pay the execution fee of the transaction\footnote{\url{http://www.reddit.com/r/ethereum/comments/2udvau/what_is_the_difference_between_gas_and_ether/}}.

The programming language to implement contracts is called \textit{Solidity}. It's a high-level turing complete programming language whose syntax is similar to JavaScript. Contracts are collections of functions and variables\footnote{\url{http://solidity.readthedocs.io/en/develop/introduction-to-smart-contracts.html}}:
\vspace{0.3cm}
\begin{lstlisting}
contract SimpleStorage {
    uint storedData;

    function set(uint x) {
        storedData = x;
    }

    function get() constant returns (uint) {
        return storedData;
    }
}
\end{lstlisting}
\vspace{0.3cm}
Mining a block on the Ethereum blockchain involves verification that exactly the deployed code was executed without tempering or inconsistencies between versions. Due to this transparency, it's easy to implement applications where normally a high level of trust between participants is needed.