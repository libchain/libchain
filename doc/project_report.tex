\documentclass[bibtotoc,liststotoc,BCOR5mm,DIV12]{scrbook}

% use this declaration to set specific page margins
%\usepackage[a4paper , lmargin = {2.7cm} , rmargin = {2.9cm} , tmargin = {2.7cm} , bmargin = {4.6cm} ]{geometry}
\usepackage[a4paper]{geometry}

\usepackage[english]{babel}
%\usepackage{bibgerm}       		% german references
\usepackage[numbers,round]{natbib}
\usepackage[utf8]{inputenc} % german characters
\usepackage{graphicx} 				% it's recommended to use PDF images but you can use JPG or PNG as well
\usepackage{url}           		% format URLs
\usepackage{hyperref} 				% create hyperlinks
\usepackage{listings, color}	% for source code
\usepackage{subfig}						% two figures next to each other (example: figure 3a), figure 3b)
\usepackage{scrpage2}					% header and footer line
\usepackage{courier}
\usepackage{float}
\restylefloat{table}
\usepackage{tabularx} % in the preamble
%\usepackage{enumitem}
\usepackage{placeins}

% header and footer line - no header & footer line on pages where a new chapter starts
\pagestyle{scrheadings}
\ohead{LibChain}

\ihead{Cabello, Janßen, Mühle}
\ofoot[]{\thepage}
\ifoot{Project Report, TU Berlin, CIT, 2017}

% set path where images are stored
\graphicspath{{./img/}}

\definecolor{mygreen}{rgb}{0,0.6,0}
\definecolor{mygray}{rgb}{0.5,0.5,0.5}
\definecolor{mymauve}{rgb}{0.58,0,0.82}

\lstset{ %
  backgroundcolor=\color{white},   % choose the background color; you must add \usepackage{color} or \usepackage{xcolor}; should come as last argument
  basicstyle=\footnotesize,        % the size of the fonts that are used for the code
  breakatwhitespace=false,         % sets if automatic breaks should only happen at whitespace
  breaklines=true,                 % sets automatic line breaking
  captionpos=b,                    % sets the caption-position to bottom
  commentstyle=\color{mygreen},    % comment style
  deletekeywords={...},            % if you want to delete keywords from the given language
  escapeinside={\%*}{*)},          % if you want to add LaTeX within your code
  extendedchars=true,              % lets you use non-ASCII characters; for 8-bits encodings only, does not work with UTF-8
  frame=single,	                   % adds a frame around the code
  keepspaces=true,                 % keeps spaces in text, useful for keeping indentation of code (possibly needs columns=flexible)
  keywordstyle=\color{blue},       % keyword style
  language=Octave,                 % the language of the code
  morekeywords={*,...},           % if you want to add more keywords to the set
  numbers=left,                    % where to put the line-numbers; possible values are (none, left, right)
  numbersep=5pt,                   % how far the line-numbers are from the code
  numberstyle=\tiny\color{mygray}, % the style that is used for the line-numbers
  rulecolor=\color{black},         % if not set, the frame-color may be changed on line-breaks within not-black text (e.g. comments (green here))
  showspaces=false,                % show spaces everywhere adding particular underscores; it overrides 'showstringspaces'
  showstringspaces=false,          % underline spaces within strings only
  showtabs=false,                  % show tabs within strings adding particular underscores
  stepnumber=2,                    % the step between two line-numbers. If it's 1, each line will be numbered
  stringstyle=\color{mymauve},     % string literal style
  tabsize=2,	                   % sets default tabsize to 2 spaces
  title=\lstname                   % show the filename of files included with \lstinputlisting; also try caption instead of title
}

\lstset{
numbers=left,
breaklines=true,
captionpos=b,
showstringspaces=false,
basicstyle={\fontfamily{pcr}\selectfont\footnotesize}}


\lstset{numbers=left,breaklines=true,captionpos=b,showstringspaces=false,
basicstyle={\fontfamily{pcr}\selectfont\footnotesize}}
\newcommand{\imp}[1]{
\begin{center}
\colorbox{red}{{#1}}
\end{center}}

\newcommand{\impp}[1]{
(\colorbox{red}{{#1}})}

\input{./misc/hyphenation} 					% use this file to set explicit hyphenations (doesn't seem to work correctly)

\begin{document}
% ---------------------------------------------------------------
\frontmatter
    \thispagestyle{empty}
\begin{center}
{\LARGE \textbf{Technische Universität Berlin}}

\vspace{0.5cm}

{\large CIT\\[1mm]}
{\large Complex and Distributed IT Systems\\[5mm]}

Faculty IV\\
Einsteinufer 17\\
10587 Berlin\\

\vspace*{1cm}
\includegraphics[width=5cm]{libchain_logo.jpg}
\vspace{0.6cm}

{\LARGE Project Report}\\
\vspace{0.3cm}
{\LARGE \textbf{LibChain - Open, Verifiable and Anonymous Access Management}}\\
\vspace*{1.0cm}
{ Juan Cabello, Gerrit Janßen and Alexander Mühle}
\\
\vspace*{0.5cm}
12.04.2017\\ % 	date of submission
\vspace*{0.5cm}

{\small submitted to}\\
Peter Janacik\\
Tim Jungnickel\\
\vspace{1.5cm}
\begin{minipage}[t]{0.33\textwidth} 
\begin{center}
\includegraphics[width=3cm]{tu_logo.jpg} 
\end{center}

\end{minipage} 
\begin{minipage}[t]{0.33\textwidth} %für den Zwischenraum 
\end{minipage} 
\begin{minipage}[t]{0.4\textwidth} 
\begin{center}
\includegraphics[width=3cm]{csm_cit-logo.jpg} 
\end{center}
\end{minipage}



\end{center}


   	\thispagestyle{empty}
%    \cleardoublepage
    
%    \include{./misc/thanx}
%    \thispagestyle{empty}
%    \cleardoublepage
    
%    \include{./misc/self-assertion} 
%    \thispagestyle{empty}
%    \cleardoublepage
    
    
%    \include{./misc/abstract}
%    \thispagestyle{empty}
%    \cleardoublepage
    
%    \include{./misc/abstract_de}
%    \thispagestyle{empty}
    
    
    \tableofcontents
    \thispagestyle{empty}
    
%    \listoffigures
%    \thispagestyle{empty}
    
%    \listoftables
%    \thispagestyle{empty}
    
% --------------------------------------------------------------



\mainmatter % comment single chapters for faster compilation

    \chapter{Blockchain Technology\label{cha:chapter1}}


\section{Roles\label{sec:ziel}}
\subsection{Publisher}
\subsection{Library}
\subsection{User}
    \chapter{LibChain}
LibChain should allows to create contracts between publisher and libraries that differ from the classical subscription models for digital media. 


\section{Architecture Overview}
The main architecture of LibChain is divided into three parts: React that is used to implement a prototypical library page, NodeJs that provides a REST interface that enables the connection to the frontend and the ethereum blockchain where the LibChain contracts are deployed on. Communication between the NodeJs backend and the blockchain are realized through the \texttt{Web3.js} api.

\vspace{0.3cm}
\includegraphics[width=\textwidth]{architecture.jpg}
\subsection{Backend}


\subsection{Frontend}


\section{Smart Contracts}
LibChain implements smart contracts for publisher, libraries and books that provides functionality to publish, buy and lend books in the LibChain ecosystem. 


\vspace{0.3cm}
\includegraphics[width=\textwidth]{contracts.png}

\subsection{LibChain}
The LibChain contract is used to instantiate new libraries and publisher. It also provides getters for retrieving existent instances.

\subsection{Book}
The Book contract collects meta-informations like title and year of publication. It also has a field \textit{gateway} that defines the access point to the digital media (for example a link to the publishers book repository). It also contains the publisher address where an instance could be bought. Furthermore it contains some variables to get books statistics, e.g. the amount of sold instances or loans.


\begin{lstlisting}
contract Book {

	address public _owner;
	string public _publisher;
	uint public _year;
	string public _gateway;
	string public _isbn;
	
	mapping (address => uint) public _balances;
	uint sumOfSoldInstances;
	uint sumOfLoans;
  
	function Book(string pub, uint year, string id, string gate) {
  	_owner = msg.sender;
		_publisher = pub;
		_year = year;
		_gateway = gate;
		_isbn = id;
	}


	function getBookInfo() constant returns (uint, string, string, 		string, uint, address, address) 
	{
	 	return (_year, _isbn, _gateway, _publisher, 	_balances[msg.sender], _owner, this);
	}

	function buy(address buyer, uint amount) {
		_balances[buyer] += amount;
		sumOfSoldInstances++;
	}

}
\end{lstlisting}

\subsection{Publisher}
The publisher contract allows to instantiate new books and add them to the publishers catalogue. It's primarily used to publish books and sell instances to libraries.

\begin{lstlisting}
function publishBook(uint year, string id, string gate) returns(address bookContract){
		publishedBooks.push(new Book(name, year, id, gate));
		sumOfPublications++;
		(..)
		return publishedBooks[bookNum-1];
	}
\end{lstlisting}

As a Book is a smart contract, it get's a unique address. Therefore, a library is able to call the \texttt{buyBook}-function to get the ownership of permanent instances.

\begin{lstlisting}
	function buyBook(address bookContract, uint amount) constant {
		Book book = Book(bookContract);
		book.buy(msg.sender, amount);	
		bills[msg.sender][bookContract] += amount;

		sumOfSoldInstances += amount;
	}
\end{lstlisting}

\subsection{Library}
The library contract contains functions to buy book instances from publisher contracts and to perfom borrow and return requests.

\paragraph*{buy}
On a buy call, the publishers buy function is called to make a note for this purchase on the publishers site. This information could be used to pay the bill in the real world.
Then the book with the requested amount of instances is stored in the libraries inventory and makes the lending possible to the library users ( see the borrow function ).

\begin{lstlisting}
	function buy(address bookContract, address publisherContract, uint amount) returns (bool) {
		Publisher pub = Publisher(publisherContract);
		pub.buyBook(bookContract, amount);
        Book book = Book(bookContract);
        if(inventory[book].amount == 0){
            inventory[book] = BookMeta(book, amount, amount);
		    _libBooks.push(book);
		} else {
            inventory[book].amount += amount;
            inventory[book].availableInstances += amount;
		}

		sumOfBoughtInstances += amount;
		return true;
	}
\end{lstlisting}


\paragraph*{borrow \label{sssec:contractborrow}}
For borrowing books, it's first check whether an instance is currently available. In addition it's ensured that a user can only borrow one instance of a book at once.

Then, the users public key is stored to the book object in the inventory. The public key is used for checking the access authorization of the user (see hasAccessToInstance function).

\begin{lstlisting}
function borrow(address bookContract, string publicKey, string userId) returns (bool) {

		if(inventory[bookContract].availableInstances <= 0) return false;

		//check if this book already loaned to user
        if (sha3(users[userId].pubkeys[bookContract]) != sha3("")){
            return false;
        }


		Book book = Book(bookContract);
        for (var i = 0; i < inventory[bookContract].amount; i++) {
            if (sha3(inventory[bookContract].pubkeys[i]) == sha3("")) {
                inventory[bookContract].pubkeys[i] = publicKey;
                inventory[bookContract].availableInstances--;

                // store loan to user object
                users[userId].loanedBooks.push(bookContract);
                users[userId].pubkeys[bookContract] = publicKey;

                book.borrow(msg.sender);
                sumOfLoans++;

                return true;
            }
        }

        return false;
	}
\end{lstlisting}

\paragraph*{Access authorization}
If a publisher gets a view-request, he can check the access authorization for this user. Here it's simply checked if a users public key is currently deposited in the book object in the library contract.
The publisher has to verify the ownership of the apropriate private key by the user on service outsite the blockchain before (see \ref{sssec:access}).

\begin{lstlisting}
	    function hasAccessToInstance(string userId, string pubkey, address bookAddress) constant returns (uint) {
        if(sha3(pubkey) == sha3("")) return 0;

        if(sha3(users[userId].pubkeys[bookAddress]) == sha3(pubkey)){
            // has access
            return 1;
        }

        // no access
        return 0;
    }
\end{lstlisting}


\paragraph*{returnBook}
To return a book, the public key entry is set to an empty string and the counter for availableInstances is increased by one.

\begin{lstlisting}
	function returnBook(address bookContract, string publicKey, string userId) returns (bool) {
    		if(inventory[bookContract].amount <= 0) return false;
    		Book book = Book(bookContract);
            for (var i = 0; i < inventory[bookContract].amount; i++) {
                if (sha3(inventory[bookContract].pubkeys[i]) == sha3(publicKey)) {
                    inventory[bookContract].pubkeys[i] = "";
                    inventory[bookContract].availableInstances++;


                    // remove book from user object
                    for (var j = 0; j < users[userId].loanedBooks.length; j++) {
                        if(users[userId].loanedBooks[i] == bookContract){
                            delete users[userId].loanedBooks[j];
                            break;
                        }
                    }
                    users[userId].pubkeys[bookContract] = "";
                    sumOfReturns;
                    return true;
                }
            }

        return false;
    }
\end{lstlisting}

\section{Use Cases}
The use case should demonstrate how to use the contract functions and on some use cases, how LibChain should be integrated in existent services of libraries and publishers.

\subsection{Book-Purchase}
\vspace{0.3cm}
\includegraphics[width=\textwidth]{purchase.png}
%\begin{itemize}
%\item a library want's to buy some book instances
%\item calling the buy method on library contract
%\end{itemize}
\subsection{Book-Borrow}
To borrow a book, the user has to authenticate toward the library service on a classical way ( e.g. username/password). The library therefore knows, that the user is authorized to lend books. If the user wants to lend a book, he has to create a RSA-keypair. The private key is stored on the client site whereas the public key is submitted togehter with the book id to the library service.
Since the user is already authenticated, the library can forward this request to the library contract without any further verification.
The contract checks the availability of the requested book instances and stores the public key (see \ref{sssec:contractborrow}).

\vspace{0.3cm}
\includegraphics[width=\textwidth]{borrow.png}


\vspace{0.3cm}
\includegraphics[width=\textwidth]{access.png}


\subsection{Book-Return}

\section{Metrics}

\section{How to implement a Publisher Service?}
In order to implement a publishing service several functions in the Ethereum Smart Contract \textit{Publisher} have to be called. \\
An instance of the Publisher contract is initiated by calling the \\
 \verb|function newPublisher(string name, string location)| of the LibChain contract. This will return an address that represents the newly created Publisher and can be used further to instantiate the Publisher. Since the LibChain contract holds a record of all registered Publishers other Libraries can now discover the newly created Publisher at any time. Additionally they can listen to the \\
 \verb|event NewPublisher(address newPublisher| to be notified of new Publishers.
\subsection{Publishing a Book}
A potential publisher can call the\\ 
 \verb|function publishBook(uint year, string id, string gate)| of their own Publisher contract. This will return an address of a new book contract and will also trigger the 
 \verb|event PublishBook(address newBook)|
\subsection{Selling a book}
Libraries that buy books from the Publisher will call the \\
\verb|function buyBook(address bookContract, uint amount)|  in this function there is a mapping from library address (the buyer) to yet another mapping which is from book address to amount. This is essentially the mapping relevant for accounting. By calling the associated getter an invoice can be created for each library and each book respectively.\\
 \verb|function getBill(address library_, address book_) constant returns (uint)|
\subsection{Access control}
If a user has borrowed a book, he can call the gateway-address that is stored in the book contract. That could be a normal internet address that routes the user to the publishers book service.
The user will then create a view request by submitting the publickey (which he has submitted to the library on the borrow request) and the book id that he encrypts with the appropriate private key.

By decrypting the book id with the submitted public key, the publisher can verify that the user is in possession of the private key. If this is successful, he can check that the public key is associated with a current loan of the book on the library by calling the \verb|function hasAccessToInstance(string pubkey, address bookAddress)| which will return 0 if access is denied and 1 respectively if it is granted. \imp{incomplete}
%    \input{./chapters/chapter3}
    \chapter{Lessons Learned}
\section{Problems and Limitations}
\section{Future Work}
%    \input{./chapters/chapter5}
%    \input{./chapters/chapter7}

% ---------------------------------------------------------------
\backmatter % no page numbering from here
%    \addchap{Liste der Akronyme}

\begin{tabbing}
spacespacespace \= space \kill
API	 \> 	Application Programming Interface \\
ORM	 \> 	Object Relational Mapper \\
OWA	 \> 	Open Web Analytics	 \\
\end{tabbing}
\endinput

		
		% if you want to provide a glossary with explanations of important terms put it in here

\bibliographystyle{abbrvdin}
\bibliography{./bib/references}

    %\renewcommand{\thesection}{\Alph{section}}
    %\addchap{Annex}

\begin{appendix}

\section{Smart Contracts \label{annex:SmartContracts}}

\lstset{language=java,numbers=left,breaklines=true,captionpos=b,showstringspaces=false,
basicstyle={\fontfamily{pcr}\selectfont\footnotesize}}
\begin{lstlisting}
    public boolean template() {
        return true;
    }

\end{lstlisting}



\end{appendix}

\endinput


\end{document}
